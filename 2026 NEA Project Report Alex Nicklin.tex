%! Author = alex-nicklin
%! Date = 10/19/25

% Preamble
\documentclass[11pt]{article}
\AddToHook{cmd/section/before}{\clearpage}
\setlength{\parindent}{0pt}

% Packages
\usepackage{amsmath}
\usepackage{graphicx}
\usepackage{fullpage}
\usepackage{listings}
\usepackage{hyperref}

\hypersetup{
    colorlinks=true,
    linkcolor=blue,
    citecolor=blue,
    filecolor=magenta,
    urlcolor=blue,
    pdftitle={2026 NEA Project Report Alex Nicklin},
    pdfpagemode=FullScreen,
}


\usepackage[backend=bibtex]{biblatex}
\addbibresource{2026NEAProjectReportAlexNicklin.bib}


% Document

\begin{document}
\begin{titlepage}
    \begin{center}

        \includegraphics[width=0.4\textwidth]{Media/Logo}

        \vspace*{1cm}

        \Huge
        \textbf{MediaFileRenamer}

        \vspace{0.5cm}
        \LARGE
        A cross-platform tool for renaming media files

        \vspace{1.5cm}

        \textbf{Alexander Nicklin}

        \Large
        Candidate number: 2444\\

        \vspace{1.5cm}

        An AQA A-Level NEA Project (7517)

        \vspace{0.8cm}

        \vspace{1.5cm}

        The Henley College, Henley-on-Thames, Oxfordshire\\
        Centre Number: 62441\\

        \vspace{0.5cm}

        19\textsuperscript{th} October 2025 \\

    \end{center}
\end{titlepage}


\tableofcontents

\section{Introduction and Declaration}\label{sec:introduction-and-declaration}

\subsection{Introduction}
This goal of this project is the write a \textbf{utility program} for renaming files. The main target will be media files on the Ubuntu operating system.

Most Windows users will use the Bulk Rename Utility\cite{BulkRenameUtility} to rename large numbers of files.
BRU is an excellent and powerful tool but it has a few problems and weaknesses. They are, in order of importance:
\begin{itemize}
    \item Not available on Linux
    \item Limited options to change media file's meta-data
    \item Not intuitive to use. A powerful utility once you understand it, but not easy for most casual users
    \item Code is closed-source
    \item Code is not free for commercial use
\end{itemize}





\textit{The design will be as simple as possible. Adding unnessary back-ends, such as a database, will not be done. If a well known and maintained library for a feature is commonly available, it will be used.}
\subsection{Declaration}
\textbf{All code - except those modules listed below - and this report is my own work.}

\textbf{No previous or related utility program has been reviewed before designing and coding this project. }

\textbf{No search for other NEA's close to this one has been done. All code and design is my solo attempt to best deliver this project and expand on the programming aspects of the course.}


Code Modules credited to other people:
\begin{itemize}
    \item spdlog\cite{spdlog} is used as a logging library
\end{itemize}

\section{Analysis}
This is the Analysis section


\section{Documented Design}
This is the Documented Design section


\section{Technical Solution}
MediaFileRenamer has been written in C++ using the QT 6\cite{QT} cross-platform UI library.
I have created build IDE's using CLion under Windows 11 and Ubuntu Desktop 25.10 to code and test this tool.
\linebreak
\linebreak
\textit{Windows installers and Ubuntu packages will be created for end users of this tool.
As this is a programming report, how these installers and packages were created is not documented here.}

\subsection{Creating a Windows Development Environment}
Follow these steps to create a Windows development environment:
\begin{itemize}
    \item Install QT from \url{https://www.qt.io/download-qt-installer-oss}
    \item Install Git from \url{https://git-scm.com/install/windows}
    \item Install CLion from \url{https://www.jetbrains.com/clion/download/?section=windows}
    \item Update all CLion Plugins
    \item Clone the MediaFileRenamer repository from \url{https://github.com/Thentoxd/MediaFileRenamer}
    \item MinGW toolchain dialog should appear - Click Next
    \item In the profile, add the following CMAKE options: \begin{verbatim}"-DCMAKE_PREFIX_PATH=C:\Qt\6.10.0\mingw_64\lib\cmake"\end{verbatim}
    \item Build files should now be created
    \item Now load up main.cpp and click Run. The project should compile and execute the program
    \item Under cmake-build-debug/logs will be the logfile generated from the program
\end{itemize}




\subsection{Creating a Linux Development Environment}

We are using the gcc compiler (version 14.2.0) on Ubuntu Desktop 25.10.


\begin{lstlisting}
chmod a+x qt-online-installer-linux-x64-4.10.0.run
sudo ./qt-online-installer-linux-x64-4.10.0.run

tar -xvzf CLion-2025.2.3.tar.gz

export Qt6_DIR=/opt/Qt/6.9.3/gcc_64

sudo apt-get install build-essential libgl1-mesa-dev
sudo apt-get install libxkbcommon-x11-dev

sudo apt install texlive-full
\end{lstlisting}

\subsection{Tested Deployment Platforms}

\subsection{Legal}
All my source code is available under the GPL licence. GPL is a copyleft license that requires all derivative works to also be released under the GPL, ensuring they remain open source

\smallskip
spdlog\cite{spdlog} is used as a logging library. Code uses the MIT licence, so can be used in both open source and proprietry programs.

\subsection{Command Line Parameters}
We are using the third-party CLI11 library to process command line parameters.

\subsection{Runtime Logging}
spdlog\cite{spdlog} is used as a logging library. This has been setup to provide one stream of messages that get sent to the console and to a three rotating 5Mb log files.
Each log sink (console and file) can be set to log all message at or above a certain threshold. The logging also produces source code and line information, eg:

\bigskip
\includegraphics[width=0.8\textwidth]{Media/2025-10-23 11_15_40 logging}

\subsection{File Metadata Processing}
https://github.com/Exiv2/exiv2, https://exiv2.org/examples.html#example1

\subsection{Config File Support}
https://github.com/nlohmann/json

\subsection{UI Developement}
\includegraphics[width=0.6\textwidth]{Media/2025-10-26 09_01_36-Screenshot First UI MediaFileRenamer}

\subsection{Documention}
This Documentation was written in Latex.
To edit and compile the Latex documentation, first:

\begin{itemize}
    \item Within CLion, install the extensions TeXiFy-IDEA and intellij-pdf-viewer
    \item Install MiKTeX from \url{https://miktex.org/download}
    \item Set MiKTeX to check and install updates automatically. Wait until it says updates are avaiable and do those
    \item I would recommend a full restart of the computer at this point
    \item Within CLion, under \begin{verbatim}\Documentation\Project Report\end{verbatim}
    \item Open 2026 NEA Project Report Alex Nicklin.tex in CLion
    \item You should be able to run this .tex file to produce a PDF
    \item This pdf will be generated in the out subdirectory
\end{itemize}

\section{Testing}

\subsection{Unit Testing}
Testing with Catch2\cite{Catch2}

\subsection{Valgrind}
\textit{One of the weakness of C and C++ is memory control. It's easy to create memory leaks and memory errors (malloc 10 bytes, write 20 etc) which can corrupt the stack and result in difficult-to-trace bugs.
To try and fix this, we use the Valgrind library to trace memory usage.}


\section{Evaluation}

This is the Evaluation section


\printbibliography

\end{document}